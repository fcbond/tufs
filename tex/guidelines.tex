%
% File acl2010.tex
%
% Contact  jshin@csie.ncnu.edu.tw or pkoehn@inf.ed.ac.uk
%%
%% Based on the style files for ACL-IJCNLP-2009, which were, in turn,
%% based on the style files for EACL-2009 and IJCNLP-2008...

%% Based on the style files for EACL 2006 by 
%%e.agirre@ehu.es or Sergi.Balari@uab.es
%% and that of ACL 08 by Joakim Nivre and Noah Smith
\PassOptionsToPackage{xetex}{xcolor}
\documentclass[11pt]{article}
\usepackage[hidelinks]{hyperref,xcolor}
\hypersetup{
     colorlinks,
     linkcolor={blue!50!black},
     citecolor={red!50!black},
     urlcolor={blue!80!black}
}
\usepackage[round]{natbib}

\usepackage{polyglossia}
\setmainlanguage{english}

%\setotherlanguage{myanmar}
%\setmainfont[Mapping=tex-text]{Times New Roman}
\usepackage{xeCJK}
%\setCJKmainfont{WenQuanYi Micro Hei}
% \setmainfont[Numbers=OldStyle]{TeX Gyre Pagella}
%\setCJKmainfont{Droid Sans Fallback}
\setCJKmainfont{Noto Sans CJK SC}
\setCJKsansfont{Noto Sans CJK SC}
%\setmainfont{Noto Sans}
\newfontfamily\urdufont[Script=Arabic,Language=Urdu,Scale=1.5]{Amiri}
% or Scheherazade after installing the font
%sudo apt-get install font-hosny-amiri
\newfontfamily\myanmarfont[Script=Myanmar]{Padauk}
\newcommand{\myanmar}[1]{{\myanmarfont #1}}
%\usepackage{acl2010}
\newcommand{\secref}[1]{\StrSubstitute{\getrefnumber{#1}}{.}{ }}
\setotherlanguage{urdu}
\setmainfont{Noto Serif}%TeX Gyre Termes}

\usepackage[j,e]{mtg2e}
\usepackage{mygb4e}
\usepackage{url}
\usepackage{latexsym}
\usepackage{graphicx}
\usepackage{enumitem}
\usepackage{omwmacros}
%\setlength\titlebox{6.5cm}    % You can expand the title box if you
% really have to

\newcommand{\lex}[1]{\textbf{\textit{#1}}}
\title{Linking the TUFS Basic Vocabulary \\ to the Open Multilingual
  Wordnet \\ Guidelines}

\author{
	Francis Bond,$^\spadesuit$ 
	Hiroki Nomoto ,$^\diamondsuit$
	Luis Morgado da Costa$^\spadesuit$\\
  $^\spadesuit$ Nanyang Technological University (NTU),  Singapore\\
  $^\diamondsuit$
  Tokyo University of Foreign Studies (TUFS), Japan \\
  \url{bond@ieee.org, nomoto@tufs.ac.jp} }

\date{}

\begin{document}
\maketitle



\section{TUFS Open Language Resources}


The data comes from the basic vocabulary words and example sentences
in 24 languages in the TUFS Open Language Resources
\citep{Kawaguchi:2007}.  These have also been used as a basis for the TUFS Asian Language
Parallel Corpus, or TALPCo \citep{}.

Nomo
This research is supported by the JSPS grant \textit{A collaborative
  network for usage-based research on lesser-studied languages}.  We
thank the creators of the individual wordnets.

\clearpage
\onecolumn
\section{Instructions for Link Checkers}

The main goal of this task is to link the concepts in the TUFS
vocabulary list with concepts in wordnet.  A straight forward example
is January: \href{14674}{the entry in tufs} matches with a
\href{http://compling.hss.ntu.edu.sg/omw/cgi-bin/wn-gridx.cgi?usrname=&gridmode=grid&synset=15210045-n&lang=eng&lang2=eng}{single
  wordnet synset}, and it matches through multiple languages.

A more typical example is \lex{dog}.  This has entries for 5 languages
in TUFS: (fr: chien; ja: 犬, ms: anjing, ur: \texturdu{کتا} and my: \myanmar{ခွေး}).  
These match to several synsets in OMW: 

\begin{figure}[htpb]
  \centering
\begin{tabular}{lcrlp{3cm}p{5cm}}
Chk &     TUFS & Score & Wordnet & Lemmas & Definition \\ \hline
1 &    犬:1.5501 &  27.58 & 02084071-n & dog; Canis familiaris;
                                           domestic dog & 	a member of
                                                          the genus Canis
                                                          (probably
                                                          descended from
                                                          the common wolf)
                                                          that has been
                                                          domesticated by
                                                          man since
                                                          prehistoric
                                                          times  \\
 0 &   犬:1.5501 &  7.76 & 03481824-n &  cock; hammer &	the part of a gunlock that strikes the percussion cap when the trigger is pulled\\
  ? &  犬 (1.5501) & 7.76 & 02084732-n & doggie; pooch; doggy; bow-wow;
                                          barker &	informal terms for dogs		\\
  0 &  犬:1.5501 &  7.76 & 07676602-n &dog; wiener; hot dog;
                                             wienerwurst; \ldots
                                             & a smooth-textured sausage of minced beef or pork usually smoked; often served on a bread roll	\\
  \end{tabular}
  
  \caption{Linking Table for 'dog'}
  \label{fig:linking}
\end{figure}


You will be given a large collection of candidate mappings (as a tab
separated file)\footnote{You should read this in with a spreadsheet,
  and sort by score and TUFS id}.  Your job
is to take the column marked Check and add '1' if it is a good match.
If it is not, you can either mark it '0' or leave it blank.  If you
mark it as '?' then it means you are not sure, and wish to discuss it
with your supervisor.    


In general, the top scoring entry (or entries) are much more likely to
be good, and the ones below less so.  If there are many matches (which
often happens) and you have one or two good matches near the top, you
can check the rest quite quickly.  Please take into account the part
of speech as well as the meaning: \textbf{n} is noun, \textbf{v} is
verb, \textbf{a} is adjective and \textbf{r} is adveRb.



We will give you an html document with all the TUFs entries in them,
with explanations, linked to the vocabulary module.  We show part of the entry
for 'dog' in Figure~\ref{fig:tufs-inu}.

\begin{figure}[htpb]
  \centering
  \includegraphics[width=0.9\textwidth]{inu-html.png}
  \caption{TUFs concept summary for 'dog' (and 'puppy')}
  \label{fig:tufs-inu}
\end{figure}

In addition to the columns shown in Figure~\ref{fig:linking}, there
are also columns for the Japanese and Chinese lemmas and definitions
(where available).  You can look up the wordnet synsets on the OMW
page:
\href{http://compling.hss.ntu.edu.sg/omw/cgi-bin/wn-grid.cgi?usrname=&gridmode=gridx&synset=02084071-n&lang=eng&lang2=eng}{entry
  for 02084071-n}.  You can cut and paste the synset ids
(e.g. 02084071-n) into the search window, or search for the lemma.


\subsection{Problematic cases}

The main problems with linking is when the granularity is not the
same.  For example, TUFS has a single entry for 切る \textit{kiru} with
two meanings listed ``switch off'' and ``cut''.    These link to very
different concepts in  wordnet.  In this case, you should link to both
concepts, and make a note: these should be two concepts in TUFS.

Sometimes the senses in wordnet are very fine: for today it has both 
15156001-n ``the day that includes the present moment (as opposed to
yesterday or tomorrow)'' and 15262921-n ``the present time or age''.
In this case you can link to both if you think both senses are likely
to be used in all languages.  Again, make a note as to which you think
is the best match.


% \subsubsection{Ambiguous Concept in TUFS}

% \subsubsection{}

\subsection{Errors in the resources}

If you find errors in either TUFS or wordnet, including the Japanese
and Chinese lemmas and definitions, please make a note.

For example, the Japanese wordnet has '洋犬' as an entry for the
synset for dog, but it should be a hyponym (foreign (Western) breed of
dog).

\end{document}

%%% Local Variables: 
%%% coding: utf-8
%%% mode: latex
%%% TeX-PDF-mode: t
%%% TeX-engine: xetex
%%% TeX-master: t
%%% End: 
